\section{Code style guide}
In this section we describe code conventions, tooling and coding workflow we intend to use for FireSale!. To start off this section uses \href{https://datatracker.ietf.org/doc/html/rfc2119}{RFC 2119} 
terms when using the phrases: \textit{must}, \textit{must not}, \textit{required}, \textit{shall}, \textit{shall not}, \textit{should}, \textit{should not}, \textit{recommended}, \textit{may}, \& \textit{optional}.\\
 
For version control we are using git, hosting the repo on GitHub, the repos \textit{main} branch is write protected to ensure proper workflow through Pull Requests (PRs). We also utilise \textit{.gitignore} to keep unwanted file out of the repo.\\

To help us keep the code-bases' indents, newlines and encoding consistent across the multiple filetypes and operating systems we use \href{https://editorconfig.org}{\textit{editorconfig}}.\\

The UTF-8 encoding and UNIX newlines are used for all files unless otherwise specified.\\

All PRs merged with the main branch will be linted for code consistency, this is done either manually or with CI if we find time to set it up.

\subsection{Comments}
Through out the code we should comment all functions and classes with some form of docstrings containing a description of what the function does along with listing of parameters and expected output.\\

Inline comments must have two full spaces before and one after the comment symbol/s followed by a short and concise comment.\\

If someone ever notices a bug but for what ever reason can´t deal with at that time it's recommended to comment 'FIXME: <short description or link to Asana>'. For functionality yet to be implemented one should comment 'TODO: <short description or link to Asana>', this makes searching for found bugs and missing functionality easier.\\

\subsection{Directory Structure}

\dirtree{%
    .1 FireSale!.
    .2 .github.
    .3 <ci>.
    .2 docs.
    .3 diagrams.
    .4 <plantUML diagrams>.
    .2 FireSale!.
    .3 <django project files>.
    .2 db.
    .3 <src>.
    .2 tests.
    .2 .editorcofnig.
    .2 .gitignore.
    .2 README.md.
}

\subsection{CSS}
Our CSS code conventions are quite straight forward, everything in a block should be lowercase, use four space soft tab for indenting, and leave a single empty line between blocks.

\subsection{HTML}
HTML files we have more points to consider than for CSS.

\begin{itemize}
	\item all tags and attributes must be in lowercase
	\item opened tags need to be closed
	\item no space around = in attribute assignment
	\item attribute values encased in double quotes
	\item images have alt, width and height defined
	\item \textbf{no inline css or js!}
\end{itemize}

\subsection{Java Script}
JavaScript is probably has the wildest style guides out there, but we decided to follow \href{https://google.github.io/styleguide/jsguide.html}{Googles JS style guide} even if it's a long winded read.

\begin{itemize}
	\item braces should be used with all keywords that support them
	\item semicolons are required
	\item when defining variables use const or let
	\item don't mix quoted and unquoted keys
	\item use JSDOC style comments for classes and functions
	\item variables and functions should be named in camelCase
	\item classes should be named in CamelCase
	\item eval evil
\end{itemize}

\subsection{Python}
For Python we plan on following \href{https://peps.python.org/pep-0008/}{PEP8} as closely one finds reasonable.\\

To give a brief summery on PEP8:
\begin{itemize}
    \item variables and functions are named in snake\_case
    \item space between all operators
    \item space after a comma not before e.g. ", "
    \item leave two empty lines between functions and/or classes, but one line between functions in the same class
    \item all functions and classes should have a docstring summarising their inner workings, parameters, and return values
    \item when defining a function use typing where it applies
\end{itemize}

\subsection{SQL}
Despite SQL not having many points in our code style it probably has the strangest.

\begin{itemize}
    \item keywords in lowercase
    \item indents should create a 'river' down the length of each given query, between keywords and parameters
    \item indent subqueries should be indented based on how far in the subquery starts
\end{itemize}

We do not expect a large portion of our codebase to be hand-crafted SQL queries but nevertheless we felt it important to at least mention some guide.